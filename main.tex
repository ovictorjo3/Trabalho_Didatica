\documentclass[12pt]{beamer}

\usetheme{Madrid}
\usecolortheme{default}

\usepackage{ragged2e}
\usepackage[utf8]{inputenc}
\usepackage[brazil]{babel}
\usepackage[T1]{fontenc}
\usepackage{amsmath}
\usepackage{amsfonts}
\usepackage{amssymb}
\usepackage{graphicx}
\usepackage{setspace}
\setbeamertemplate{caption}[numbered]
\usepackage{hyperref}
\setbeamercovered{transparent} 
\setbeamertemplate{navigation symbols}{}
\usepackage[table,xcdraw]{xcolor}
\usepackage{ragged2e}

%referência e citação

\usepackage[backend=biber]{biblatex}
\addbibresource{ref.bib}



\addtobeamertemplate{block alerted}{}{\justifying}

\author[Sarraff \and Joao Victor \and ...]{% 
Sterphanlliny Sarraff\inst{1} \and Joao Victor\inst{2} \and Marcos Maurício\inst{2} \\ \and Kauã Vidal\inst{2} \and Everton Souza\inst{2}}
\title[Prática docente: primeira reflexão]{Prática docente: primeira reflexão}
\institute[]{% 
  \textsuperscript{1} Licenciatura em Física \\
  \textsuperscript{2} Licenciatura em Matemática 
}

\titlegraphic{\includegraphics[width=0.2\textwidth]{imagens/logo_3.jpg}}
\date{\today} 


%\subject{}

% ---------------------------------------------------------

\begin{document}
\onehalfspacing 
\justifying 

\begin{frame}
    \titlepage
\end{frame}

\begin{frame}{Sumário}
    \tableofcontents
\end{frame}

\section{Ensinar exige rigorosidade metódica}
\section{Ensinar exige pesquisa}
\section{Ensinar exige respeito aos saberes dos educandos}
\section{Ensinar exige criticidade}

	\begin{frame}{Tópico 1.4 - Ensinar exige criticidade} 
		
		\textbf{Ensinar é promover o pensamento crítico e a autonomia intelectual.}
		
		\begin{itemize}
			\justifying
			\item Superação da curiosidade ingênua pela curiosidade epistemológica.
			\item Pensar certo: refletir sobre a realidade com profundidade.
			\item O professor também aprende ao ensinar.
			\item A criticidade é essencial para transformar a sociedade.
		\end{itemize}
		
	\end{frame}
	
	\begin{frame}{Tópico 1.4 - Implicações pedagógicas}
		
		\begin{itemize}
			\justifying
			\item Estimular o questionamento e a reflexão. \\
			\item Valorizar o saber do senso comum como ponto de partida. \\
			\item Promover o diálogo e a escuta ativa. \\
			\item Formar sujeitos críticos e conscientes. \\
		\end{itemize}
		
	\end{frame}
	
\section{Ensinar exige estética e ética}

	\begin{frame}{Tópico 1.5 - Ensinar exige estética e ética}
		
		\begin{alertblock}{\centering A educação deve ser guiada por valores éticos e sensibilidade estética}
			
			\justifying
			\textbf{(1)} Ética: respeito, justiça, coerência e responsabilidade. \\
			\uncover<2->{\textbf{(2)} Estética: beleza, cuidado, sensibilidade no ato de ensinar.}\\
			\uncover<3->{\textbf{(3)} A prática docente deve ser humanizada e significativa.} \\
			\uncover<4->{\textbf{(4)} A forma de ensinar também comunica valores.} \\
		\end{alertblock}
		
	\end{frame}
	
	\begin{frame}{Tópico 1.5 - Implicações pedagógicas}
			\justifying
			\textbf{(1)} Criar ambientes acolhedores e inspiradores. \\
			\uncover<2->{\textbf{(2)} Ser exemplo de ética na relação com os alunos.}\\
			\uncover<3->{\textbf{(3)} Valorizar a linguagem, o espaço e os gestos.} \\
			\uncover<4->{\textbf{(4)} Ensinar com amorosidade e respeito à dignidade humana.} \\
	\end{frame}
	
	\begin{frame}{Uma conexão dos tópicos}
		
		\begin{alertblock}{\justifying A criticidade forma o pensamento livre e consciente.}
			
			\justifying
			\textbf{(1)} A ética orienta o compromisso com o outro. \\
			\uncover<2->{\textbf{(2)} A estética dá beleza e sentido à prática educativa.}\\
			\uncover<3->{\textbf{(3)} Juntas, essas dimensões constroem uma educação transformadora.} \\
			
		\end{alertblock}
		
	\end{frame}
	
	\begin{frame}{Conclusão}
		
		\begin{alertblock}{\justifying Ensinar é um ato político, ético e estético.}
			
			\justifying
			\textbf{(1)} O professor deve ser crítico, sensível e comprometido. \\
			\textbf{(2)} A educação libertadora exige mais que técnica: exige humanidade.\\
	
		\end{alertblock}
		
	\end{frame}
\section{Ensinar exige a corporificação das palavras pelo exemplo}
\section{Ensinar exige risco, aceitação do novo e rejeição a qualquer forma de discriminação}
\section{Ensinar exige reflexão crítica sobre a prática}

	\begin{frame}{Tópico 1.8 - Ensinar exige reflexão crítica sobre a prática}
		\begin{enumerate}
			\justifying
			\item A prática docente do fazer e do pensar sobre o fazer; \\
			\item O saber ingênuo da prática docente; \\
			\item \textit{"É pensando criticamente a prática de hoje ou de ontem que se pode melhorar a próxima prática".} (FREIRE, 2006, p. 33)
		\end{enumerate}
	\end{frame}
	
\section{Ensinar exige o reconhecimento e a assunção da identidade cultural}

	\begin{frame}{Tópico 1.9 - Ensinar exige o reconhecimento e a assunção da identidade cultural}
		\begin{enumerate}
			\justifying
			\item A reflexão sobre a \textit{assunção}; \\
			\item A importância dos gestos no espaço escolar; \\
			\item \textit{"Não há prática docente verdadeira que não seja ela mesma um ensaio estético e ético, permita-se-me a repetição".}  (FREIRE, 2006, p. 37)
			
		\end{enumerate}
	\end{frame}
	
	\begin{frame}{Reflexão}
		\justifying
		
		\textit{"O gesto do professor valeu mais do que a própria
			nota dez que atribuiu à minha redação. O gesto do
			professor me trazia uma confiança ainda obviamente desconfiada de que era possível trabalhar e produzir. De que era possível confiar em mim, mas que seria tão errado confiar além dos
			limites quanto errado estava sendo não confiar"} \\ \ (FREIRE, 2006, p. 43)
	\end{frame}

\section{Referências}

\begin{frame}{Referências}
	
	\begin{itemize}
		\justifying
		\item FREIRE, P. \textbf{Pedagogia da Autonomia: saberes necessários à prática educativa}. São Paulo : Paz e Terra, 2011.
		
		\item PAULA, Marcelo de. \textbf{Estudo Sobre a Pedagogia da Autonomia de Paulo Freire}.[recurso eletrônico]. Porto Alegre, RS: Editora Fi, 2016. 
		
		\item MATIAS, Carlos dos Passos Paulo. \textbf{Pedagogia da autonomia: saberes necessários à prática educativa}. Criar Educação, v. 5, n. 2, 2016.
		
		\item SILVA, Luísa Helena et al. \textbf{Pedagogia da autonomia: saberes necessários à prática educativa}. Nucleus, v. 16, n. 2, p. 97-100, 2019.
		
	\end{itemize}
\end{frame}

\end{document}